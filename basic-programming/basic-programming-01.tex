\documentclass{article}
\usepackage{geometry}
\usepackage{hyperref}
\usepackage{listings}
\usepackage{xcolor}
\usepackage{amsmath}

\geometry{margin=1in}

% Define Rust code listing style
\lstdefinelanguage{Rust}{
	keywords={fn, let, mut, pub, use, mod, struct, enum, impl, for, while, loop, if, else, match, return, break, continue},
	keywordstyle=\color{blue}\bfseries,
	ndkeywords={u8, i32, bool, String, Result, Option},
	ndkeywordstyle=\color{purple}\bfseries,
	comment=[l]{//},
	commentstyle=\color{gray}\ttfamily,
	stringstyle=\color{red}\ttfamily,
	sensitive=true
}

\lstset{
	language=Rust,
	basicstyle=\ttfamily\small,
	columns=flexible,
	numbers=left,
	numberstyle=\tiny\color{gray},
	stepnumber=1,
	numbersep=5pt,
	showspaces=false,
	showstringspaces=false,
	tabsize=4,
	breaklines=true,
	breakatwhitespace=true,
	frame=single,
	backgroundcolor=\color{white},
	captionpos=b
}

\title{Programming 101: Introduction to Rust with D\&D Character Building}
\author{}
\date{}

\begin{document}
	
	\maketitle
	
	\section*{Objective}
	This document introduces you to basic programming concepts in Rust while creating a Dungeons \& Dragons (D\&D) character. You will learn to work with variables, data types, and basic operations in Rust by building elements of a D\&D character sheet.
	
	\section*{What is a Variable?}
	A \textbf{variable} is like a container that holds a value. Think of it as a labeled box where you can store information about your character, such as their name, class, or strength.
	
	\subsection*{Declaring Variables}
	In Rust, you use the \texttt{let} keyword to declare variables. When declaring a variable, you can specify its type, such as a whole number for an attribute like strength or a text string for the character’s name.
	
	\begin{lstlisting}[language=Rust]
		let strength: u8 = 15;  // Declares an integer variable named strength
		let name: &str = "Arin"; // Declares a string variable named name
	\end{lstlisting}
	
	\section*{Basic Data Types in Rust}
	In Rust, you’ll use data types to specify what kind of information a variable holds. Here are a few common types we’ll use for our D\&D character:
	
	\begin{itemize}
		\item \textbf{Integer (\texttt{u8})}: Represents a whole number (from 0 to 255), ideal for scores like strength or intelligence.
		\item \textbf{Boolean (\texttt{bool})}: Represents a true or false value, which could be used for states like \texttt{is\_alive}.
		\item \textbf{String (\texttt{\&str})}: Represents text, useful for names, classes, and descriptions.
	\end{itemize}
	
	\section*{Task: Create Basic Character Attributes}
	In this task, you will declare variables for some basic character attributes. Define the following variables:
	\begin{itemize}
		\item \texttt{name}: A string holding your character’s name.
		\item \texttt{class}: A string representing your character’s class (e.g., "Wizard", "Fighter").
		\item \texttt{strength}, \texttt{dexterity}, \texttt{intelligence}: Integers holding attribute scores between 1 and 20.
	\end{itemize}
	
	\subsection*{Example Code}
	Here’s how you might declare these variables in Rust:
	
	\begin{lstlisting}[language=Rust]
		fn main() {
			let name: &str = "Arin";
			let class: &str = "Fighter";
			let strength: u8 = 16;
			let dexterity: u8 = 12;
			let intelligence: u8 = 14;
			
			println!("Character Name: {}", name);
			println!("Class: {}", class);
			println!("Strength: {}", strength);
			println!("Dexterity: {}", dexterity);
			println!("Intelligence: {}", intelligence);
		}
	\end{lstlisting}
	
	\section*{Performing Basic Operations}
	In D\&D, each attribute score has a modifier. A modifier is calculated by the formula:
	\[
	\text{modifier} = \frac{\text{score} - 10}{2}
	\]
	We’ll use Rust to calculate the modifier for each attribute.
	
	\subsection*{Task: Calculate Modifiers}
	Using the formula above, write a function that calculates the modifier for a given attribute score.
	
	\begin{lstlisting}[language=Rust]
		fn calculate_modifier(score: u8) -> i8 {
			((score as i8 - 10) / 2)
		}
		
		fn main() {
			let strength: u8 = 16;
			let dexterity: u8 = 12;
			let intelligence: u8 = 14;
			
			println!("Strength Modifier: {}", calculate_modifier(strength));
			println!("Dexterity Modifier: {}", calculate_modifier(dexterity));
			println!("Intelligence Modifier: {}", calculate_modifier(intelligence));
		}
	\end{lstlisting}
	
	\section*{Testing Your Code}
	To test your code, make sure you’re in the correct project directory and run the following command:
	\begin{lstlisting}[language=bash]
		cargo run
	\end{lstlisting}
	You should see output similar to:
	\begin{verbatim}
		Character Name: Arin
		Class: Fighter
		Strength: 16
		Dexterity: 12
		Intelligence: 14
		Strength Modifier: 3
		Dexterity Modifier: 1
		Intelligence Modifier: 2
	\end{verbatim}
	
	\section*{Additional Exercises}
	Now that you understand variables and basic operations, try the following exercises:
	\begin{itemize}
		\item Change the character’s class and attribute scores. Rerun the program to see how modifiers change.
		\item Add more attributes, such as \texttt{wisdom} and \texttt{charisma}, and calculate their modifiers.
		\item Modify the code to display a short character description based on their class and attributes.
	\end{itemize}
	
	\section*{Next Steps}
	Once you’re comfortable declaring variables and performing calculations, you’ll be ready to move on to more complex tasks, such as querying additional character information from the D\&D API. Each task will build on your knowledge to create a richer character experience.
	
\end{document}
