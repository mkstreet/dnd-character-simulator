\documentclass{article}
\usepackage{geometry}
\usepackage{hyperref}
\usepackage{listings}
\usepackage{xcolor}

\geometry{margin=1in}

% Define Rust code listing style
\lstdefinelanguage{Rust}{
	keywords={fn, let, mut, pub, use, mod, struct, enum, impl, for, while, loop, if, else, match, return, break, continue},
	keywordstyle=\color{blue}\bfseries,
	ndkeywords={u8, i32, bool, String, Result, Option},
	ndkeywordstyle=\color{purple}\bfseries,
	comment=[l]{//},
	commentstyle=\color{gray}\ttfamily,
	stringstyle=\color{red}\ttfamily,
	sensitive=true
}

\lstset{
	language=Rust,
	basicstyle=\ttfamily\small,
	columns=flexible,
	numbers=left,
	numberstyle=\tiny\color{gray},
	stepnumber=1,
	numbersep=5pt,
	showspaces=false,
	showstringspaces=false,
	tabsize=4,
	breaklines=true,
	breakatwhitespace=true,
	frame=single,
	backgroundcolor=\color{white},
	captionpos=b
}

\title{Programming 101: Introduction to Rust with D\&D Characters}
\author{}
\date{}

\begin{document}
	
	\maketitle
	
	\section*{Welcome to Programming 101}
	This document introduces basic programming concepts using Rust and Dungeons \& Dragons (D\&D) characters. You'll learn to declare variables, understand data types, and use functions to interact with a simulated character sheet.
	
	\section*{What is a Variable?}
	A \textbf{variable} is a way to store information in a program. Think of it as a labeled box where you can keep different values, like your character’s strength or name.
	
	\subsection*{Declaring Variables}
	When you \textbf{declare} a variable, you create it and give it a name. In Rust, use the \texttt{let} keyword to declare variables:
	\begin{lstlisting}[language=Rust]
		let strength: u8 = 15; // Declares an integer variable named strength
	\end{lstlisting}
	
	\section*{Basic Data Types in Rust}
	Here are common data types used in Rust:
	\begin{itemize}
		\item \textbf{Integer (\texttt{u8})}: A whole number (0-255). Use for attributes like strength.
		\item \textbf{String (\texttt{\&str})}: Text, like a character’s name or class.
		\item \textbf{Boolean (\texttt{bool})}: A true or false value, like whether a character is currently alive.
	\end{itemize}
	
	\section*{Navigating the Project Directory}
	As you work on different tasks, you’ll need to navigate between project directories:
	\begin{itemize}
		\item \texttt{cd <directory>}: Change to a specific directory.
		\item \texttt{pwd}: Print the current working directory.
		\item \texttt{cargo run}: Run a Rust program from the current project.
	\end{itemize}
	
	\subsection*{Switching Between Projects}
	To access various tasks:
	\begin{enumerate}
		\item **Basic programming tasks** are in \texttt{basic-programming/basic\_syntax}:
		\begin{lstlisting}[language=bash]
			cd ~/workspaces/dnd-character-simulator/basic-programming/basic_syntax
		\end{lstlisting}
		
		\item **D\&D character tasks** are in \texttt{src}:
		\begin{lstlisting}[language=bash]
			cd ~/workspaces/dnd-character-simulator/src
		\end{lstlisting}
	\end{enumerate}
	
	\section*{Interacting with the D\&D API}
	Using Rust, you can query your character’s attributes through specific functions:
	\begin{itemize}
		\item \texttt{get\_character\_name()}: Returns a character’s name.
		\item \texttt{get\_strength()}: Returns the strength score.
		\item \texttt{get\_intelligence()}: Returns the intelligence score.
	\end{itemize}
	
	\section*{Running Rust Programs}
	Once you’re in the correct project directory, use this command to run the program:
	\begin{lstlisting}[language=bash]
		cargo run
	\end{lstlisting}
	
	\section*{Basic Workflow Tips}
	\begin{itemize}
		\item Use \texttt{pwd} to check your current directory.
		\item Save your code regularly.
		\item Use \texttt{git add}, \texttt{git commit}, and \texttt{git push} to save changes to GitHub.
	\end{itemize}
	
	\section*{Next Steps}
	Once you’re comfortable navigating directories and running Rust programs, you’re ready to start working with the D\&D character tasks. Check each task document for detailed instructions.
	
\end{document}
