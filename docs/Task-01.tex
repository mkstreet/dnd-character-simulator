\documentclass{article}
\usepackage{geometry}
\usepackage{listings}
\usepackage{xcolor}
\usepackage{amsmath}

\geometry{margin=1in}

% Define Rust code listing style and set it as the default language
\lstdefinelanguage{Rust}{
	keywords={fn, let, pub, use, mod, struct, impl},
	keywordstyle=\color{blue}\bfseries,
	ndkeywords={u8, String},
	ndkeywordstyle=\color{purple}\bfseries,
	comment=[l]{//},
	commentstyle=\color{gray}\ttfamily,
	stringstyle=\color{red}\ttfamily,
	sensitive=true
}

\lstset{
	language=Rust,
	basicstyle=\ttfamily\small,
	columns=flexible,
	numbers=left,
	numberstyle=\tiny\color{gray},
	stepnumber=1,
	numbersep=5pt,
	showspaces=false,
	showstringspaces=false,
	tabsize=4,
	breaklines=true,
	breakatwhitespace=true,
	frame=single,
	backgroundcolor=\color{white},
	captionpos=b
}

\title{Task 01: Retrieve Basic Character Information}
\author{}
\date{}

\begin{document}
	
	\maketitle
	
	\section*{Objective}
	In this task, you will retrieve and display basic information about your D\&D character using the API functions. This task will introduce you to Rust syntax for function calls and printing values to the console.
	
	\section*{Instructions}
	
	1. **Call the API Functions**: Use the functions provided in the D\&D Character Simulator API to retrieve character details.
	2. **Store Values in Variables**: Assign the retrieved values to variables so that you can use them later in your code.
	3. **Print the Character Information**: Display the character’s name, class, ancestry, and a couple of attributes in the terminal.
	
	\subsection*{Example Code}
	Here’s some example code to help you get started. This code retrieves and prints the character’s name, class, ancestry, and a few attributes.
	
	\begin{lstlisting}
		fn main() {
			let name = get_character_name();
			let class = get_character_class();
			let ancestry = get_character_ancestry();
			let strength = get_strength();
			let intelligence = get_intelligence();
			
			println!("Character Name: {}", name);
			println!("Class: {}", class);
			println!("Ancestry: {}", ancestry);
			println!("Strength: {}", strength);
			println!("Intelligence: {}", intelligence);
		}
	\end{lstlisting}
	
	\section*{Running the Code}
	To test your code, make sure you’re in the correct project directory. Run the following command in the terminal to see the output:
	\begin{lstlisting}[language=bash]
		cargo run
	\end{lstlisting}
	
	You should see output similar to:
	\begin{verbatim}
		Character Name: Arin
		Class: Fighter
		Ancestry: Elf
		Strength: 16
		Intelligence: 14
	\end{verbatim}
	
	\section*{Additional Exercise}
	After successfully displaying the basic information, try the following:
	\begin{itemize}
		\item Retrieve and print additional attributes, such as \texttt{dexterity} and \texttt{charisma}.
		\item Write a function to calculate and print the modifier for each attribute.
	\end{itemize}
	
	\section*{Next Steps}
	This task introduces basic function calls and printing in Rust. Once you’re comfortable with these steps, you’ll be ready to explore more advanced interactions with the D\&D Character API.
	
\end{document}
